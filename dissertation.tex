%%% PREAMBLE

\documentclass[11pt]{report} % This is in order to have the chapter feature and page numbers in the same location on even and odd pages.
% Acceptable font sizes are 10 to 12 pts

\usepackage[papersize={8.5 in, 11 in}, nohead, includeheadfoot, left=1.5 in, right = 1 in, vmargin= 1 in]{geometry}
% The options above are to set the paper size (as per page 11 of the guidelines and the margins. Includefoot is to make sure that nothing
% gets printed on the margins. The statutory margins are set on page 5.

\usepackage[doublespacing]{setspace} % Needed to set double-spacing for the main document, but needs more adhoc commands to
								% use single spacing for tables, etc.
% This is a helpful package that puts math inside length specifications
\usepackage{calc}

% For the signature boxes and much better looking tables
\usepackage{array} % Also for better arrays (eg matrices) in maths
\usepackage{tabularx}
\usepackage{booktabs}

% Commands to format the Table of Contents, List of Tables and List of Illustrations, as per the Wharton Template
\setcounter{tocdepth}{2}

\usepackage[toctextentriesindented]{tocstyle}
\usetocstyle{standard}

\usepackage[toc,page]{appendix}

%\usepackage{tocloft}
%\cftsetindents{chapter}{0em}{1.5em}

% \usepackage{remreset} % Remove counters from reset list


% Commands (more further down, at preliminary, main and appendix) to change the formatting of chapter headings
\usepackage[compact]{titlesec}
%\renewcommand{\beforetitleunit}{0 pt}
%\titleformat{\section}[hang]{\large}{\thesection.}{6 pt}{}
%\titleformat{\subsection}[hang]{\normalsize\itshape}{\thesubsection.}{6 pt}{}
%\titlespacing*{\section}{0pt}{0pt}{0pt}
%\titlespacing*{\subsection}{0pt}{0pt}{0pt}

%\usepackage{parskip} % To allow for better management of the Dutch paragraph style
\usepackage{url} % To allow for better typing of the url in the Creative Commons part of the copyright
\usepackage[numbers,sort&compress]{natbib} % allows the author-date citation system
%\bibpunct{(}{)}{;}{a}{,}{,} % options for natbib to yield the citation style I like

% Other packages that are not needed for the template, but I highly recommend (and all of your own packages go here to)
% Add them one by one to make sure they do not interfere, for instance the package subfigure clashes with this template
\usepackage{amssymb}
\usepackage{amsfonts}
\usepackage[cmex10]{amsmath}
\usepackage{amsthm}
\newtheorem{theorem}{Theorem}
\newtheorem{remark}{Remark}

\usepackage{graphicx}
\DeclareGraphicsExtensions{.pdf,.jpeg,.jpg,.png,.PNG,.JPG}

%\usepackage[section]{placeins} % defines a \FloatBarrier command beyond which floats may not pass. A package option allows you to declare that floats may not pass a \section command, but you can place \FloatBarriers wherever you choose.
%\usepackage{morefloats}

\usepackage{longtable}
\usepackage{dcolumn}
\usepackage{tabularx}

\usepackage{rotating}
\usepackage{pdflscape}

\usepackage{xtab}
\usepackage{paralist} % very flexible & customizable lists (eg. enumerate/itemize, etc.)
\usepackage{verbatim}
%\usepackage{subfiles} % To be better able to manage large projects by compiling the separate files included in the final document

\usepackage[T1]{fontenc}
\usepackage{siunitx}

\usepackage[font=small]{caption}
\usepackage[font=small]{subfig}

%\usepackage{cite}

%\usepackage[pagebackref=true]{hyperref}
\usepackage[bookmarks=true]{hyperref}
\hypersetup{
  pdfborder={0 0 0},
  colorlinks=true,
  linkcolor=black,    % color of internal links
  citecolor=black,    % color of links to bibliography
  filecolor=black,    % color of links to files
  urlcolor=black      % color of urls
}
\usepackage[figure,table]{hypcap} % so that hyperref links jump to top of figures

\usepackage{amssymb}
\usepackage[mathscr]{eucal} % adds \mathscr command for Euler script, else overwrites mathcal
\usepackage{nicefrac}

\usepackage{algorithm}
\usepackage{algpseudocode}

\usepackage{array}


% *** FLOAT PACKAGES ***
%\usepackage{fixltx2e}
% fixltx2e, the successor to the earlier\end{landscape} fix2col.sty, was written by
% Frank Mittelbach and David Carlisle. This package corrects a few problems
% in the LaTeX2e kernel, the most notable of which is that in current
% LaTeX2e releases, the ordering of single and double column floats is not
% guaranteed to be preserved. Thus, an unpatched LaTeX2e can allow a
% single column figure to be placed prior to an earlier double column
% figure. The latest version and documentation can be found at:
% http://www.ctan.org/tex-archive/macros/latex/base/


%\usepackage{stfloats}
% stfloats.sty was written by Sigitas Tolusis. This package gives LaTeX2e
% the ability to do double column floats at the bottom of the page as well
% as the top. (e.g., "\begin{figure*}[!b]" is not normally possible in
% LaTeX2e). It also provides a command:
%\fnbelowfloat
% to enable the placement of footnotes below bottom floats (the standard
% LaTeX2e kernel puts them above bottom floats). This is an invasive package
% which rewrites many portions of the LaTeX2e float routines. It may not work
% with other packages that modify the LaTeX2e float routines. The latest
% version and documentation can be obtained at:
% http://www.ctan.org/tex-archive/macros/latex/contrib/sttools/
% Do not use the stfloats baselinefloat ability as IEEE does not allow
% \baselineskip to stretch. Authors submitting work to the IEEE should note
% that IEEE rarely uses double column equations and that authors should try
% to avoid such use. Do not be tempted to use the cuted.sty or midfloat.sty
% packages (also by Sigitas Tolusis) as IEEE does not format its papers in
% such ways.
% Do not attempt to use stfloats with fixltx2e as they are incompatible.
% Instead, use Morten Hogholm'a dblfloatfix which combines the features
% of both fixltx2e and stfloats:
%
% \usepackage{dblfloatfix}
% The latest version can be found at:
% http://www.ctan.org/tex-archive/macros/latex/contrib/dblfloatfix/




%\ifCLASSOPTIONcaptionsoff
%  \usepackage[nomarkers]{endfloat}
% \let\MYoriglatexcaption\caption
% \renewcommand{\caption}[2][\relax]{\MYoriglatexcaption[#2]{#2}}
%\fi
% endfloat.sty was written by James Darrell McCauley, Jeff Goldberg and
% Axel Sommerfeldt. This package may be useful when used in conjunction with
% IEEEtran.cls'  captionsoff option. Some IEEE journals/societies require that
% submissions have lists of figures/tables at the end of the paper and that
% figures/tables without any captions are placed on a page by themselves at
% the end of the document. If needed, the draftcls IEEEtran class option or
% \CLASSINPUTbaselinestretch interface can be used to increase the line
% spacing as well. Be sure and use the nomarkers option of endfloat to
% prevent endfloat from "marking" where the figures would have been placed
% in the text. The two hack lines of code above are a slight modification of
% that suggested by in the endfloat docs (section 8.4.1) to ensure that
% the full captions always appear in the list of figures/tables - even if
% the user used the short optional argument of \caption[]{}.
% IEEE papers do not typically make use of \caption[]'s optional argument,
% so this should not be an issue. A similar trick can be used to disable
% captions of packages such as subfig.sty that lack options to turn off
% the subcaptions:
% For subfig.sty:
% \let\MYorigsubfloat\subfloat
% \renewcommand{\subfloat}[2][\relax]{\MYorigsubfloat[]{#2}}
% However, the above trick will not work if both optional arguments of
% the \subfloat command are used. Furthermore, there needs to be a
% description of each subfigure *somewhere* and endfloat does not add
% subfigure captions to its list of figures. Thus, the best approach is to
% avoid the use of subfigure captions (many IEEE journals avoid them anyway)
% and instead reference/explain all the subfigures within the main caption.
% The latest version of endfloat.sty and its documentation can obtained at:
% http://www.ctan.org/tex-archive/macros/latex/contrib/endfloat/
%
% The IEEEtran \ifCLASSOPTIONcaptionsoff conditional can also be used
% later in the document, say, to conditionally put the References on a
% page by themselves.





% *** CUSTOM MACROS ***
\newcommand{\ie}{\emph{i.e.},\ }
\newcommand{\eg}{\emph{e.g.},\ } % the \ ensures latex does not treat command as end of sentence

\newcommand{\innerprod}[2]{\ensuremath{\left\langle #1, \, #2 \right\rangle}}
\newcommand{\E}[1]{\mathbf{E}\left[#1\right]} % expectation
\newcommand{\1}[2]{\mathbf{1}_{#2}\left( #1 \right)}

\renewcommand{\P}{\mathbb{P}}

\newcommand{\bb}{\mathbf{b}}
\newcommand{\be}{\mathbf{e}}
\newcommand{\bbf}{\mathbf{f}}
\newcommand{\bq}{\mathbf{q}}
\newcommand{\br}{\mathbf{r}}
\newcommand{\bs}{\mathbf{s}}
\newcommand{\bu}{\mathbf{u}}
\newcommand{\bv}{\mathbf{v}}
\newcommand{\bx}{\mathbf{x}}
\newcommand{\by}{\mathbf{y}}
\newcommand{\bz}{\mathbf{z}}

\newcommand{\bomega}{\boldsymbol{\omega}}

\newcommand{\Q}{\mathcal{Q}}
\newcommand{\bigO}{\mathcal{O}}
\newcommand{\G}{\mathcal{G}}

\renewcommand{\H}{\mathscr{H}}
\newcommand{\V}{\mathscr{V}}
\newcommand{\W}{\mathscr{W}}
\newcommand{\X}{\mathscr{X}}
\newcommand{\Y}{\mathscr{Y}}
\newcommand{\Z}{\mathscr{Z}}

\newcommand{\R}{\ensuremath{\mathbb{R}}}
\newcommand{\I}{ \{ 0,1 \} }

\newcommand{\pf}{p_{\rm f}}
\newcommand{\ps}{p_{\rm s}}
\newcommand{\pfp}{p_{\rm fp}}
\newcommand{\pfn}{p_{\rm fn}}

\newcommand{\qf}{\overline{p}_{\rm f}}
\newcommand{\qs}{\overline{p}_{\rm s}}
\newcommand{\qfp}{\overline{p}_{\rm fp}}
\newcommand{\qfn}{\overline{p}_{\rm fn}}

\DeclareMathOperator*{\argmax}{argmax}
\DeclareMathOperator*{\argmin}{argmin}

\newcommand{\chapref}[1]{\hyperref[#1]{Chapter~\ref*{#1}}}
\newcommand{\appendixref}[1]{\hyperref[#1]{Appendix~\ref*{#1}}}
\newcommand{\secref}[1]{\hyperref[#1]{Section~\ref*{#1}}}
\newcommand{\figref}[1]{\hyperref[#1]{Figure~\ref*{#1}}}
\newcommand{\tabref}[1]{\hyperref[#1]{Table~\ref*{#1}}}
\newcommand{\algoref}[1]{\hyperref[#1]{Algorithm~\ref*{#1}}}

%\newcommand{\blankline}[1][1]{\quad\vspace{(1 + (\real{#1} - 1))*\baselineskip}\pagebreak[3]}
\newcommand{\blankline}{\quad\pagebreak[3]}
\newcommand{\halfblankline}{\quad\vspace{-0.5\baselineskip}\pagebreak[3]}

\newcommand*{\CDot}{\scalebox{0.6}{\textbullet}}%\raisebox{-0.25ex}{\scalebox{1.2}{$\cdot$}}}

\newtheorem{lemma}{Lemma}

\theoremstyle{definition}
\newtheorem{defn}{Definition}
\newtheorem{issue}{Issue}
\newtheorem{rmrk}{Remark}

%%% BEGIN DOCUMENT

\begin{document}
%\SIunits[amssymb]

% Making tables and figures numbered continuously
% \makeatletter
% \@removefromreset{table}{chapter}
% \makeatother
% \renewcommand{\thetable}{\arabic{table}}
% \makeatletter
% \@removefromreset{figure}{chapter}
% \makeatother
% \renewcommand{\thefigure}{\arabic{figure}}

%\def\sym#1{\ifmmode^{#1}\else\(^{#1}\)\fi} % Defining stars for significance tables

% Defining variables to be used throughout the document for personalization
\def\mytitle{Title of the Thesis} % Make sure this is in all caps
\def\myauthor{Author Name}
\def\myauthorfull{Author Long Name}
\def\mysupervisorname{Advisor Name}
\def\mysupervisortitle{Advisor Title}
\newlength{\superlen}   % a "scratch" length
\settowidth{\superlen}{\mysupervisorname, \mysupervisortitle} % Width of signature line for supervisor
\def\gradchairname{Grad Chair Name}
\def\gradchairtitle{Grad Chair Title}
\newlength{\chairlen}   % a "scratch" length
\settowidth{\chairlen}{\gradchairname, \gradchairtitle} % Width of signature line for supervisor
\newlength{\maxlen}
\setlength{\maxlen}{\maxof{\superlen}{\chairlen}}
\def\mydepartment{Mechanical Engineering and Applied Mechanics}
\def\myyear{20XX}
\def\signatures{36 pt} % Space to accommodate the signatures, you can fiddle with this as you like

%% PRELIMINARY PAGES

\pagenumbering{roman}
\pagestyle{plain}

% TITLE PAGE

\begin{titlepage}
\thispagestyle{empty} % No page numbers on title page, as per Manual page 8
\begin{center}

%\onehalfspacing

\mytitle

\myauthor

A DISSERTATION

in 

\mydepartment 

Presented to the Faculties of the University of Pennsylvania

in 

Partial Fulfillment of the Requirements for the

Degree of Doctor of Philosophy

\myyear

% Added for proposal
\vfill

The Dissertation proposal will take place:

Month XX, 20XX 

XX:XX PM

Building, Room XXX

\vfill
% end proposal
\end{center}

\vfill % Here to make sure the page is filled


\begin{flushleft}

%\blankline
%
%\blankline
%
%\singlespacing
%
%\rule{\linewidth}{0.3mm}
%\mysupervisorname, Supervisor of Dissertation \\
%\mysupervisortitle
%
%\doublespacing
%
%\blankline
%
%\blankline
%
%\singlespacing
%
%\rule{\linewidth}{0.3mm}
%\gradchairname, Graduate Group Chairperson \\
%\gradchairtitle
%
%\doublespacing
%
%\halfblankline

\singlespacing

Dissertation Committee % No signature necessary

\blankline

Member One, Title of Member One

Member Two, Title of Member Two

Member Three, Title of Member Three

\end{flushleft}

\end{titlepage}

% COPYRIGHT NOTICE (optional)

\doublespacing

\thispagestyle{empty} % No page number as per Manual, p. 11

% Changing formatting for preliminary pages (NOT OPTIONAL)

% ABSTRACT
\begingroup
\titleformat{\chapter}[hang]{\large\center}{\thechapter}{0 pt}{}
\titlespacing*{\chapter}{0pt}{-33 pt}{6 pt} % The key value here is the -33 pts, I got to it by old fashioned measuring with a ruler....

\clearpage
%\onehalfspacing
\chapter*{ABSTRACT}
\addcontentsline{toc}{chapter}{Abstract} % This is to include this section in the Table of Contents
\begin{center}
\mytitle

\myauthor

\mysupervisorname

\end{center}

Text of abstract goes here.
Text of abstract goes here.
Text of abstract goes here.
Text of abstract goes here.
Text of abstract goes here.
Text of abstract goes here.
Text of abstract goes here.
Text of abstract goes here.
Text of abstract goes here.
Text of abstract goes here.

\endgroup

% TABLE OF CONTENTS

\clearpage
\singlespacing

\phantomsection % To make hyperref jump to the correct location
\addcontentsline{toc}{chapter}{Contents}
\tableofcontents

% LIST OF TABLES

\clearpage
\phantomsection % To make hyperref jump to the correct location
\addcontentsline{toc}{chapter}{List of Tables}
\listoftables

% LIST OF ILLUSTRATIONS

\clearpage
\phantomsection % To make hyperref jump to the correct location
\addcontentsline{toc}{chapter}{List of Figures}
\listoffigures

% PREFACE (OPTIONAL)

%\clearpage
\doublespacing
%\chapter*{Preface}
%\addcontentsline{toc}{chapter}{Preface} % This is to include this section in the Table of Contents

%\endgroup

%% MAIN TEXT
\newpage
\pagenumbering{arabic}
\pagestyle{plain} % This has to be repeated here because the lists change the style

% Dutch style of paragraph formatting, i.e. no indents.
%\setlength{\parskip}{10 pt} % Same as Word file
%\setlength{\parindent}{0pt}

\chapter{Introduction}
\label{chap:introduction}

Text of introduction chapter goes here \cite{bibtex_key}.
\chapter{Title of First Chapter of Content}
\label{chap:one}

Text of first chapter of content goes here.
\chapter{Title of Second Chapter of Content}
\label{chap:two}

Text of second chapter of content goes here.
\chapter{Conclusion}
\label{chap:conclusion}

Text of conclusion chapter goes here.

\begin{appendices}

%%%%%%%%%%%%%%%%%%%%%%%%%%%%%%%%%%%%%%%%%%%%%%%%%%%%%%%%%%%
\chapter{Title of First Appendix Chapter}
\label{sec:appendix:alpha}

Text of first appendix chapter goes here.
%%%%%%%%%%%%%%%%%%%%%%%%%%%%%%%%%%%%%%%%%%%%%%%%%%%%%%%%%%%
\chapter{Title of Second Appendix Chapter}
\label{sec:appendix:beta}
%%%%%%%%%%%%%%%%%%%%%%%%%%%%%%%%%%%%%%%%%%%%%%%%%%%%%%%%%%%

Text of second appendix chapter goes here.

\end{appendices}


%%% BIBLIOGRAPHY

\clearpage
\singlespacing

\renewcommand\bibname{Bibliography}
\bibliographystyle{plainnat}
%\bibliographystyle{plain}
\bibliography{IEEEfull,references} % This is the filename of the bibtex bibliography file (it has to be in the same directory as the main LaTeX file)
\addcontentsline{toc}{chapter}{Bibliography}


%% BIBLIOGRAPHY
\clearpage
\singlespacing
\phantomsection % To make hyperref jump to the correct location
\addcontentsline{toc}{chapter}{Bibliography}
\renewcommand\bibname{Bibliography}
\bibliographystyle{plainnat}
%\bibliographystyle{plain}
\bibliography{IEEEfull,references} % This is the filename of the bibtex bibliography file (it has to be in the same directory as the main LaTeX file)


%% INDEX (OPTIONAL) - I do not really know how to code this, sorry

\end{document}
